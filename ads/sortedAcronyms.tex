%		\ac{Abk.}   --> fügt die Abkürzung ein, beim ersten Aufruf wird zusätzlich automatisch die ausgeschriebene Version davor eingefügt bzw. in einer Fußnote (hierfür muss in header.tex \usepackage[printonlyused,footnote]{acronym} stehen) dargestellt
%		\acf{Abk.}   --> fügt die Abkürzung UND die Erklärung ein
%		\acl{Abk.}   --> fügt nur die Erklärung ein
%		\acp{Abk.}  --> gibt Plural aus (angefügtes 's'); das zusätzliche 'p' funktioniert auch bei obigen Befehlen
%		\acs{Abk.}   -->  fügt die Abkürzung ein
%	siehe auch: http://golatex.de/wiki/%5Cacronym
%!TEX root = ../dokumentation.tex
%nur verwendete Akronyme werden letztlich im Abkürzungsverzeichnis des Dokuments angezeigt
%Verwendung: 
\acro{API}{Application Programming Interface}
\acro{bpm}{Beats per minute}
\acro{CSV}{Comma-separated Values}
\acro{EKG}{Elektrokardiogramm}
\acro{FFT}{Fast-Fourier-Transformation}
\acro{Hf}{Herzfrequenz}
\acro{HF}{High Frequency}
\acro{HRV}{Herzfrequenzvariabilität}
\acro{HV}{Herzminutenvolumen}
\acro{LF}{Low Frequency}
\acro{OLS}{Ordinary least squares}
\acro{PTSD}{Post Traumatic Stress Disorder}
\acro{RMSSD}{Root Mean Square of successive differences}
\acro{SADS}{Sudden arrhythmic death syndrome}
\acro{SDNN}{Standard Derivation of NN-Intervals}
\acro{SUDEP}{Sudden unexpected death in epilepsy}
\acro{SV}{Schlagvolumen}
\acro{ULF}{Ultra Low Frequency}
\acro{VLF}{Very Low Frequency}
