%!TEX root = ../dokumentation.tex

\chapter{Aufgabenstellung}

Mit der fortschreitenden Entwicklung in der Digitalisierung unserer Gesellschaft erhöht sich ebenfalls die Anzahl und Verfügbarkeit notwendiger Geräte und der dazugehörigen Sendeanlagen. Diese können mit Hilfe von elektromagnetischen Wellen kabellos Daten übertragen und erzeugen somit künstlich elektromagnetische Felder, welche sich mit den bereits natürlichen elektromagnetischen Feldern überlagern. Die Folgen des wachsenden Ausbaus digitaler kabelloser Technologien, sowie die Nutzung der passenden Endgeräte, wie Smartphones, Tablets und drahtlose Telefone und die damit verbundenen Auswirkung auf die Bevölkerung, rückte bisher nur in geringem Maße in den Fokus der wissenschaftlichen Forschung.\\
Innerhalb einer Pilotstudie sollen die Auswirkungen und Einflüsse elektromagnetischer Strahlung auf die Stress- und Reizverarbeitung des Menschen untersucht werden. Dazu werden verschiedene biologische Messwerte des Herzens ohne Strahlenbelastung bestimmt und danach den alltäglichen Belastungen durch WLAN-Router, DECT-Telefone, Bluetooth- und Mobilfunk-Endgeräte gegenübergestellt. Bei der Durchführung der verschiedenen Messungen und Untersuchungsarten der Studie wird eine große Datenmenge produziert, welche passend aufbereitet werden muss, um eine schnelle und einfache Auswertung durch die beteiligten Ärzte und Physiotherapeuten ermöglichen zu können.

Ziel der Studienarbeit ist die Konzeptionierung und Implementierung einer Schnittstelle, welche die Automatisierung der Aufbereitung und Auswertung der Messdaten ermöglicht. Dazu soll eine Applikation erstellt werden, welche an die bereits verwendete und etablierte Software Kubios HRV Premium anknüpft und deren Funktionen erweitert. So sollen die Rohdaten der Messungen in Kubios eingelesen und die Exportdateien und deren Herzparameter im nächsten Schritt innerhalb der Applikation graphisch dargestellt werden. Der Fokus liegt dabei auf der automatische Generierung der in Kubios erstellbaren Samples, welche die Einteilung der Messung in kleinere Zeitbereiche ermöglicht, sowie der fachlich korrekten Darstellung der Herzparameter im Diagramm zur Unterstützung der Studie. Zudem soll die Möglichkeit bestehen das Belastungsintervall hervorzuheben und die fertige Visualisierung zu speichern. Zur Umsetzung sollen dabei ebenfalls unterschiedliche Programmiersprachen sowie Dateiformate betrachtet und bewertet werden. 