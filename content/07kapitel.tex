%!TEX root = ../dokumentation.tex

\chapter{Kritische Würdigung und Ausblick}
Im Bezug auf die Entwicklung eines Tools zur Auswertung medizinischer Messdaten lässt sich zusammenfassend sagen, dass die formulierten Funktionen und Anforderungen nahezu problemlos umgesetzt werden konnten. Das Tool ermöglicht das Auswerten und Visualisieren von medizinischen Messdaten. Zudem können Samples konfiguriert und Messungen passend exportiert werden. Auch die fachlich korrekte Darstellung aller Messdaten mit einem Belastungsintervall, welches zusätzlich hervorgehoben werden kann, sind im Rahmen der Arbeit vollständig implementiert worden.\\
Besonders hervorgehoben werden muss auch die Benutzerfreundlichkeit der Applikation, sowie die Möglichkeit diese mit Hilfe einer Installer-Datei sowohl auf Windows als auch auf MacOS zu verteilen und zu installieren.

Abweichungen bezüglich des Konzepts treten hauptsächlich bei der automatischen Generierung von Samples in Kubios. Aufgrund der eingeschränkten Schnittstellen der Software, konnte keine vollständige Automatisierung implementiert werden, weshalb das Exportieren der mit Samples versehenen Messung noch manuell ausgeführt werden muss. Dieses Problem ist jedoch vollständig Kubios verschuldet und kann möglicherweise durch zukünftige Versionen der Software durch eine Kommandozeilen-Schnittstelle behoben werden.\\
Auch das Design der Applikation wurde über den Zeitraum der Entwicklung immer wieder verändern und angepasst, um diese möglichst übersichtlich sowie intuitiv gestalten und zusätzliche Funktionen passend in die Benutzeroberfläche integrieren zu können.

Die Entwicklung innerhalb von MATLAB erwies sich als sehr praktisch und intuitiv, was sich vor allem auf den nativen MATLAB-Datentyp des Messungsexport, sowie das schnelle Erstellen einer Benutzeroberfläche durch den graphisch unterstützte App Designer zurückführen lässt. Außerdem konnte die große Zahl an hilfreicher und umfassender Dokumentation als wichtiger Teil der Problemlösung kennengelernt werden.

Für die Zukunft und der Weiterentwicklung sollte vor allem Fokus auf die Anpassbarkeit der Applikation gelegt werden. So sollte eine Funktion implementiert werden, die es dem Benutzer ermöglicht die Setup-Datei direkt innerhalb des Tools zu bearbeiten, um die Erweiterung durch weitere Parameter sowie die Anpassung auf eine mögliche andere Datenstruktur des Exports von Kubios ermöglichen zu können.\\
Ein weiterer Punkt ist die Speicherung einer Sitzung. Im Moment geht die erstellte Visualisierung und die geladene Messung, sowie jegliche Einstellungen beim Beenden der Applikation verloren. Ein übergreifender Speichermechanismus könnte dieses Problem lösen und möglicherweise auch eine passende Grundstruktur zur Speicherung und Anpassung des Setups ermöglichen.\\
Zuletzt kann davon ausgegangen werden, dass sich einige fehlende Funktionen oder benötigte Anpassung zukünftig vor allem bei der Anwendung durch die Beteiligten der Pilotstudie herausarbeiten lassen.