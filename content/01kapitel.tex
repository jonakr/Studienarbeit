%!TEX root = ../dokumentation.tex

\chapter{Einleitung}
\todo{tbd..}
\section{lorem ipsum}
\subsection{merol muspi}
\subsubsection{Tim riecht}
\Blindtext
\begin{equation}\formelentry{Beispielformel}
%a = b + \lambda - \frac{\phi - \lambda}{2 \cdot \pi}
\acs{a} = \acs{b} + \acs{lambda} - \frac{\acs{phi} - \acs{lambda}}{2 \cdot \pi}
\end{equation} 
\subsection{ipsum lorem}
\blindtext

Ein Beispiel für die Verwendung eines Acronyms \ac{BSP}
mit Verweis auf eine Quelle \cite{Wollschlaeger2014}.
Dies ist ein Beispiel für einen \gls{Glossareintrag}.

\begin{figure}[!htbp]
    \centering
    \includegraphics[width=0.3\linewidth]{dhbw_de}
    \caption{DHBW-Logo}
    \label{fig:dhbw_logo}
\end{figure}

\section{Listings}
\subsection{lorem listum}

\begin{lstlisting}[caption={Einbinden von Code aus externer Datei mit Angabe eines Zeilenbereichs},label=inputFromFile]
\lstinputlisting[language=Python,firstline=37,lastline=45]{source_filename.py}
\end{lstlisting}

\blindmathpaper

\begin{lstlisting}[caption=Dies ist ein Listing,label=lstcode]
for i:=maxint to 0 do
begin
{ do nothing }
end;
Write('Case insensitive ');
Write('Pascal keywords.');
\end{lstlisting}

\begin{figure}[ht]%
	\centering
	\begin{circuitikz}
		\draw (0,0)
				to[V,v=$U_q$] (0,2)
				to[R=$R_1$] (2,2)
				to[short] (6,2)
				to[R=$R_2$] (6,0)
				to[short] (0,0);
	\end{circuitikz}
	\caption{Elektrische Schaltung - ein Beispiel}%
\end{figure}
